% =============================================================================

\subsection{Compatibility}
\label{sec:bg:compatibility}

\begin{itemize}

\item \XCRYPTO is a 
      non-standard extension~\cite[Section 21.1]{SCARV:RV:ISA:I:17},
      in the sense it {\em may} conflict with other 
      (non-standard {\em or} standard)
      extensions.  In particular, note that

      \begin{itemize}
      \item the 
            instruction encodings
            do not have a domain-separating prefix, i.e., \XCRYPTO is a
            brownfield extension~\cite[Section 21.1]{SCARV:RV:ISA:I:17}
            wrt. the base ISA,
            whereas
      \item the 
            instruction mnemonics 
            do     have a domain-separating prefix:
            for example, \VERB[RV]{foo} is captured as \VERB[RV]{xc.foo}.
      \end{itemize}

\item Per the current list\footnote{
      See, e.g., \url{http://workspace.riscv.org}
      } of (public) RISC-V working groups, 
      \XCRYPTO relates to elements of (at least) the following:

      \begin{enumerate}
      \item Cryptographic Extensions Task Group,
      \item BitManip                      Group,
            and
      \item P             Extension  Task Group (i.e., the embedded DSP-like ISE).
      \end{enumerate}

      \noindent
      In some cases overlap exists, potentially representing an opportunity
      for unification; in other cases the underlying goal, approach, and/or
      ethos is distinct and thus incompatible.

\end{itemize}

% =============================================================================
