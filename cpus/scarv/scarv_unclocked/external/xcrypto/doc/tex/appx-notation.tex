% =============================================================================

\subsection{General}

\begin{itemize}

\item $x \ASN     y$
      is used to denote an assignment of the value $y$
      to the variable $x$.
\item $x \RAND    Y$
      is used to denote an assignment of a   value $y$,
      sampled uniformly at from the set $Y$ st. $y \in Y$,
      to the variable $x$.
\item $x \TEST{=} y$
      is used to denote a  test or comparison,
      in this case equality,
      between $x$ and $y$ that yields a Boolean (i.e., \TRUE or \FALSE) result.
\item $x \RANGE   y$
      denotes the range of values between $x$ and $y$ inclusive; this implies
      a need to generate intermediate values, st. the notation is only useful
      iff. doing so is obvious {\em and} unambiguous.

\item  $\CARD{X}$
      denotes the cardinality (or size) of some object $X$.
\item $\INDEX{X}{i}$
      refers to an indexed (or numbered)
      element   within some object     $X$, namely the $i$-th such element.
\item $\FIELD{X}{Y}$
      refers to a  named (or labelled)
      field $Y$ within some object     $X$.
\item $\SCOPE{X}{Y}$
      refers to a  named (or labelled)
      field $Y$ within some name-space $X$;
      this is useful to disambiguate different $Y$ with the same identifier.

\end{itemize}

% -----------------------------------------------------------------------------

\subsection{Numbers}

\begin{itemize}

\item $\RADIX{x}{b}$ denotes $x$ is expressed in radix- or base-$b$; where no
      base is specified, it is safe to assume decimal (i.e., that $b = 10$).
\item $\CARD{\RADIX{x}{b}}$ denotes the number of digits in $x$, if expressed
      in base-$b$.
\item Unless otherwise stated, a little-endian digit ordering is assumed; the
      least-significant (resp. most-significant) digit is thus $\INDEX{x}{0}$
      (resp. $\INDEX{x}{n}$).

\end{itemize}

% -----------------------------------------------------------------------------

\subsection{Bits}

\begin{itemize}

\item The following terms refer to fixed-length (sub-)sequences of $n$ bits
      \[
      \begin{array}{lcc@{\;}c@{\;}r}
      \mbox      {crumb} &\Longrightarrow& n &=&  2 \\
      \mbox     {nibble} &\Longrightarrow& n &=&  4 \\
      \mbox       {byte} &\Longrightarrow& n &=&  8 \\
      \mbox  {half-word} &\Longrightarrow& n &=& 16 \\
      \mbox       {word} &\Longrightarrow& n &=& 32 \\
      \mbox{double-word} &\Longrightarrow& n &=& 64 \\
      \end{array}
      \]
      where perhaps all but the first are fairly standard.
\item Where appropriate, a sequence $x$ of $n$ bits can be interpreted as an
      element of the polynomial ring $\B{F}_2[\IND{x}]$
      (i.e. a binary polynomial);
      the $i$-bit of $x$ basically represents the $i$-th coefficient, st.
      $
      x \mapsto x( \IND{x} )
      $
      because
      $
      x_i \mapsto x_i \cdot \IND{x}^i .
      $
\item The operators $\NOT$, $\AND$, $\IOR$, and $\XOR$ denote Boolean NOT, AND,
      OR, and XOR respectively, with $\NAND$, $\NIOR$ and $\NXOR$ denoting NAND,
      NOR, and NXOR; all of these may be overloaded to cater for bit-sequences
      rather than simply bits.
\item $\HW( X )$
      denotes the Hamming weight   of      bit-sequence  $X$,         st.
      \[
      \HW( X )    = \sum_{i=0}^{i<n} \INDEX{X}{i}                   .
      \]
\item $\HD( X )$
      denotes the Hamming distance between bit-sequences $X$ and $Y$, st.
      \[
      \HD( X, Y ) = \sum_{i=0}^{i<n} \INDEX{X}{i} \XOR \INDEX{Y}{i} .
      \]
\item Given a bit-sequence
      \[
      X = \LIST{ \INDEX{X}{n-1}, \ldots, \INDEX{X}{1}, \INDEX{X}{0} } 
      \]
      and assuming $w$ divides $\CARD{X}$, $\INDEX[w]{X}{i}$ is used to denote 
      the $i$-th sub-sequence of length $w$ in $X$: for $n = 16$, for example, 
      we have that
      \[
      \begin{array}{lcl}
      \INDEX[4]{X}{0} &=& \LIST{ \INDEX{X}{ 3}, \INDEX{X}{ 2}, \INDEX{X}{ 1}, \INDEX{X}{ 0}                                                             } \\
      \INDEX[4]{X}{2} &=& \LIST{ \INDEX{X}{11}, \INDEX{X}{10}, \INDEX{X}{ 9}, \INDEX{X}{ 8}                                                             } \\
      \INDEX[8]{X}{0} &=& \LIST{ \INDEX{X}{ 7}, \INDEX{X}{ 6}, \INDEX{X}{ 5}, \INDEX{X}{ 4}, \INDEX{X}{ 3}, \INDEX{X}{ 2}, \INDEX{X}{ 1}, \INDEX{X}{ 0} } \\
      \INDEX[8]{X}{1} &=& \LIST{ \INDEX{X}{15}, \INDEX{X}{14}, \INDEX{X}{13}, \INDEX{X}{12}, \INDEX{X}{11}, \INDEX{X}{10}, \INDEX{X}{ 9}, \INDEX{X}{ 8} } \\
      \end{array}
      \]
      for example.  In essence, this is a short-hand st.
      \[
      \INDEX[w]{X}{i} \equiv \LIST{ \INDEX[w]{X}{w \cdot i}, \INDEX[w]{X}{w \cdot i + 1}, \ldots, \INDEX[w]{X}{w \cdot i + w - 1} } .
      \]
\item $\LSB[l]{X}$ (resp. $\MSB[l]{X}$) denotes the $l$ least- (resp. most-)
      significant bits of $X$ (where we assume $l = 1$ if omitted): given a
      bit-sequence
      \[
      X = \LIST{ \INDEX{X}{n-1}, \ldots, \INDEX{X}{1}, \INDEX{X}{0} } ,
      \]
      for example, and assuming $l \leq \CARD{X}$, we have that
      \[
      \begin{array}{lcl}
      \LSB[l]{X} &=& \LIST{ \INDEX{X}{l-1},                 \ldots, \INDEX{X}{  1}, \INDEX{X}{  0} } \\
      \MSB[l]{X} &=& \LIST{ \INDEX{X}{n-1}, \INDEX{X}{n-2}, \ldots                  \INDEX{X}{n-l} } \\
      \end{array}
      \]

\end{itemize}

% -----------------------------------------------------------------------------

\subsection{Operations}

\begin{itemize}

\item The operators $\LSH$ and $\RSH$ denote left- and right-shift; $\LRT$ and
      $\RRT$ denote left- and right-rotate (beware of context: $\ll$ and $\gg$
      are also used to denote ``much less than'' and ``much greater than'').
\item The operators 
      $\oplus$
      and
      $\otimes$
      respectively denote addition and multiplication in $\B{F}_2[\IND{x}]$.
\item For some $x$, 
      we let
      $\EXT[w]{0  }( x )$
      and
      $\EXT[w]{\pm}( x )$
      respectively denote
      zero- or sign-extension to $w$-bits
      (allowing omission of either specifier where appropriate).
\item For some operator $\odot$, 
      we let
      $\OP[w][s]{\odot}$
      and 
      $\OP[w][u]{\odot}$
      respectively denote 
      $w$-bit signed and unsigned variants
      (allowing omission of either specifier where appropriate).

\end{itemize}

% -----------------------------------------------------------------------------

\subsection{Cryptography}

\begin{itemize}

\item $\ROUND{x}{k}$ denotes some quantity $x$ relating to the $k$-th 
      round (e.g., the $k$-th round key).
\item The AES round functions are denoted using
      $\SCOPE{\ID{AES}}{\ALG{Key-Addition}}$,
      $\SCOPE{\ID{AES}}{\ALG{Sub-Bytes}}$,
      $\SCOPE{\ID{AES}}{\ALG{Shift-Rows}}$,
      and
      $\SCOPE{\ID{AES}}{\ALG{Mix-Columns}}$;
      note that 
      a) $\SCOPE{\ID{AES}}{\ALG{Mix-Column}}$ is applied per column of the
         state matrix to realise $\SCOPE{\ID{AES}}{\ALG{Mix-Columns}}$, so
         the former is basically a $1$-column version of the latter,
      b) $\SCOPE{\ID{AES}}{\ALG{S-Box}}$ denotes the AES S-box,
         and
      c) $\SCOPE{\ID{AES}}{\ALG{X}}^{-1}$ denotes the inverse of $\ALG{X}$,
         e.g., $\SCOPE{\ID{AES}}{\ALG{S-Box}}^{-1}$ is the inverse S-box.

\end{itemize}

% -----------------------------------------------------------------------------

\subsection{Architecture}

\begin{itemize}

\item $\RNG$
      denotes the Random Number Generator (RNG) object;
      we equip this object with three operations, namely
      \[
      \begin{array}{c@{\;}c@{\;}ccl}
      x    &\TEST{\ASN}& \RNG &:& \mbox{test                    the RNG state} \\
      x    &      \ASN & \RNG &:& \mbox{sample entropy $x$ from the RNG      } \\
      \RNG &      \ASN & x    &:& \mbox{inject entropy $x$ into the RNG      } \\
      \end{array}
      \]
\item $\MEM$
      denotes the byte-addressable memory;
      $\MEM[*][i]$ 
      denotes the $i$-th,
            $8$-bit 
      entry in said memory.
\item $\GPR$ 
      denotes the 
      general-purpose register file;
      $\GPR[*][i]$ 
      denotes the $i$-th,
      $\RVXLEN$-bit
      entry in said register file.

\end{itemize}

% =============================================================================
