\subsection{\XCRYPTO exceptions}
\label{sec:spec:exceptions}

% =============================================================================

Any instruction-specific exception behaviour is captured by the associated
semantics in \REFSEC{sec:spec:instruction}, with general cases captured in
the following:

\begin{itemize}
\item Any attempt to execute an \XCRYPTO instruction which
      has a    valid opcode but at least one invalid operand,
      will raise an 
      illegal instruction exception.
\item Any attempt to execute a
        valid \XCRYPTO instruction
      from an unsupported feature class (or parameterisation thereof, cf. packed operations, per fields in $\SPR{mxcsr}$)
      will raise an 
      illegal instruction exception.
\item Any instruction that accesses $\GPR$ using a register address $i$ 
      will raise an
      illegal instruction exception
      if $i \geq |\GPR|$; for example
      RV32I sets $|\GPR| = 32$ so demands $0 \leq i < 32$,
      whereas
      RV32E sets $|\GPR| = 16$ so demands $0 \leq i < 16$.
\item Any access to
      $\SPR{mxcsr}$
      from a privilege mode other than machine mode
      will raise an 
      illegal instruction exception.
\item Any instruction that accesses memory mirrors RV32I wrt. exception,
      synchronisation, and atomicity semantics: for example, if
      a) the effective address stemming from a memory access instruction
         is not aligned to the associated data type, 
         {\em  and}
      b) the implementation does not support misaligned accesses,
         {\em then} it 
      will raise a 
      load/store address misaligned exception.
\end{itemize}

% =============================================================================
