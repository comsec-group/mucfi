% =============================================================================

\subsection{Assumptions}
\label{sec:bg:assumption}

\begin{itemize}

\item We make {\em no} assumption about the value of $\RVXLEN$ as defined
      by the base ISA.  That is, \XCRYPTO can cater for instances where
      $
      \RVXLEN \in \SET{ 32, 64, 128 } ,
      $
      although, since it has a focus on micro-controller class host cores
      per \REFSEC{sec:bg:concept},
      $
      \RVXLEN = 32
      $
      is the norm.

\item \XCRYPTO demands interaction with an RNG, the concrete instantiation of 
      which is unspecified: we assume the RNG design follows best-practice,
      e.g., per NIST~\cite{SCARV:NIST:SP:800_90a,SCARV:NIST:SP:800_90b,SCARV:NIST:SP:800_90c},
      and has an interface per \cite[Section 6.4]{SCARV:NIST:SP:800_90c}.

      On one hand, doing so affords flexibility in an implementation; this 
      is important, in that an RNG selection and implementation will likely 
      be technology-specific (e.g., differ for a given FPGA, vs. an ASIC).  
      On the other hand, however, the RNG used is critically important wrt. 
      security: the (difficult) challenge of selecting and implementing 
      a robust RNG instance is assumed to be addressed.

\end{itemize}

% =============================================================================
